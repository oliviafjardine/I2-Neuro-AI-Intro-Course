\documentclass[letterpaper,11pt]{article}

\usepackage[shortlabels]{enumitem}
\usepackage[margin=1in]{geometry}
\usepackage[most]{tcolorbox}

\begin{document}
\title{{\bf Unit 3: Basic Neuroanatomy Project} }
\author{Name: Olivia Jardine}

\date{}
\maketitle

\section*{Short Answer}
\begin{enumerate}[a)]
\item Describe several advantages and disadvantages of biological computation with the brain compared to machine learning

\begin{tcolorbox}
In biological computation within the brain, there are billions of neurons that all connect in complex ways. An advantage is that its very adaptable to certain situations. Given a set of information, it's able to make intuitive decisions, generalize knowledge, and adapt to changing environments. In comparison, machine learning is less flexible, using less creativity and intuition. ML is better with large-scale datasets and can process and recognize patterns faster. The disadvantage to ML is that it carries a high dependency on large amounts of data to learn effectively.
\end{tcolorbox}

\item Speculate what aspects of the architecture of the brain may cause these advantages or disadvantages, and similarly comment on aspects of machine learning’s architecture

\begin{tcolorbox}
The interconnected neurons within the brain communicate through complex electrochemical signals. These systems allow information to be processed in parallel. The connections between neurons, synapses, can be modified by experience and learning, hence its adaptability. This is neuroplasticity. In machine learning, artificial neural networks require suitable hardward like processors, storage, and memory. Its learning comes from trial and error, so new data may lead to poor generalization.
\end{tcolorbox} 

\item Brainstorm some marvelous schemes for integrating advantages from both ways of computing. Draw, write, scribble etc… When you are done, do a quick google for your best ideas to see if anyone has researched or tried them already!

\begin{tcolorbox}
Neuromorphic computing combines both biological computation and machine learning with the purpose to mimic neuro-biological architectures in the brain and nervous system. An example of this is Intel's Hala Point, the largest neuromorphic computer. However, the challenge with this lies within the immense complexity of the biological computing as the human brain contains over 86 billion neurons, each with an average of 7000 synapses. Doing the math, that's 600 trillion connections. We have only covered a fraction of that. As of right now, the complexity is past our capabilities. Though maybe we can emulate smaller models of biological neural networks geared for certain functions/tasks.
\end{tcolorbox}
    
\end{enumerate}
\end{document}